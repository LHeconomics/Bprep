% Table 3
%\documentclass[8pt, letterpaper]{article}
%\begin{document}
\hypertarget{Potential problems and solutions at patient, nurse, physician and system level for the bowel preparation process for inpatient colonoscopy}{}
 \begin{flushleft}
 \begin{table}
  \caption{Potential problems and solutions at patient, nurse, physician and system level for the bowel preparation process for inpatient colonoscopy}
  \label{table:problems}
    \begin{tiny}
    \begin{tabular}{lll} \hline
    \textbf{Level} & \textbf{Potential Problems} & \textbf{Potential Solutions}\\ \hline
    Patient & Poor palatability & Switch to Polyethylene glycol-electrolyte solution with flavor\\
     & & \ \ packets of 4 flavors attached to bottle and no sulfur taste\\
     & & Chill bowel preparation \\
     & & Mix with flavor powders or non-red juice \\
     & Unable to drink fast enough due to volume or nausea & Anti-emetics\\
     & \ \ (assuming obstruction not suspected) or altered mental status & \\
     & \ \ (i.e. delirium or dementia) & \\
     & & Nasogastric Tube\\
     & Not following instructions & Patient education handout \\
     & & Family involvement\\
     & Inpatient status $\pm$ underlying risk factors for poor bowel preparation & Consider 6-L bowel preparation \\
     & & Consider 2-day bowel preparation\\
     & Flushes bowel movement before nurse evaluating & Nursing places toilet hat when close to ready \\
     & & \\
    Nursing & Floor nurse protocol cannot require bedside checks more frequently & Encourage family to help \\
     & \ than every 4 h & \\
     & & Recruit medical assistant participation \\
     & & Use of technology for reminders\\
     & Unclear importance of bowel preparation and instructions & Standardize instructions\\
     & \ highly variable & \\
     & & Nursing education sessions\\
     & & Endoscopy and floor nurses discuss day prior to procedure\\
     & Original nurse communication with instructions acknowledged by & Instructions in medication order so viewable when administering\\ 
     & \ day shift nurse and not viewed by night shift nurse & Orderset with timed instructions \\
     & Variable reporting of readiness for procedure & Nursing education and picture of readiness on patient education\\
     & & Toilet hat\\
     & & Endoscopy and floor nurses discuss morning of procedure\\
     & & \\
    Physician & Preparation recommended by gastroenterology highly variable & Create protocol for standardization \\
     & \ \ leading to confusion & \\
     & Instructions from gastroenterology not clear and/or written in notes & Electronic note templates for easy use in gastroenterology notes\\
     & Due to nature of complex inpatient consult service, decision-to- & Set mutually agreed upon expectation for early communication\\
     & \ \ scope communicated late (i.e., after 6 pm) to primary team & \ \ by gastroenterology with a set latest time (i.e., 4 pm)\\
     & Ordering suppositories and enemas as ``rescue'' in the morning leads & Using more bowel preparation instead of suppositories\\
     & \ \ to false sense patient is clear when right side of colon is not & \ \ and enemas \\
     & & Conversion to 2-day preparation \\
     & Boston Bowel Preparation Score not properly documented. This & Scoring education \\
     & \ \ could be knowledge gap or due to busy inpatient consult service & Document score in Brief-Op note immediately post-procedure for \\
     & \ \ while scoping. Procedure notes written at end of day leading to & \ \ reference later when writing procedure note\\
     & \ \ memory and bias & \\
     & Primary team orders differently than gastroenterology & Primary team education \\
     & recommendations & Orderset \\
     & & \\
    System & Amount of bowel preparation consumed not documented & Fellow or nurse go to bedside to document amount drank\\
     & & Educate nursing day prior to document in medical record\\
     & Lag time between order and administration & Stock bowel preparation in pyxis on specific floors\\
     & Long chain of communication: GI, primary team, day nurse, & Set protocol for communication expectations, note templates, \\
     & \ \ night nurse & \ \ ordersets \\
     & Dietary keeps flavor mix packs and nursing unable to get after & Stock flavor packs on floor or use Polyethylene glycol-electrolyte \\
     & \ \ certain hour & \ \ solution with flavor packets of 4 flavors attached to bottle \\ \hline
    \end{tabular}
    \end{tiny}
 \end{table}
 \end{flushleft}
%\end{document} 